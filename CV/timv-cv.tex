\documentclass{report}

\usepackage{graphicx}
\usepackage{url}

\usepackage[utf8]{inputenc}
\usepackage[T1]{fontenc}

\usepackage{parskip}
\setlength{\parskip}{10pt}

\setlength{\textwidth}{6.0in}
\setlength{\textheight}{9.5in}
\oddsidemargin .25in
\evensidemargin 1in
\topmargin -0.5in


\newcommand{\centerheader}[1]{\moveleft.5\hoffset\centerline{#1}}

\begin{document}


{\vskip 0.1in}

{
  {\vskip 0.2in}
  \centerheader{\Large \bf Timothy F. Vieira}
  {\par
    {\vskip 0.1in}
    \centerheader{\texttt{tim.f.vieira@gmail.com}}
    \centerheader{\url{http://timvieira.github.com/}}
  }
  {\vskip 0.1in}
}
{\vskip -0.20in}


\section*{Education} 
{\bf PhD} \emph{in progress} (Fall 2011-current) Johns Hopkins University \\
{\sl Advised by:} Jason Eisner (\url{http://cs.jhu.edu/~jason})

{\bf Bachelor of Science} University of Illinois Urbana-Champaign \\
{\sl Major:} Computer Science, College of Engineering \\
{\sl Concentration:} Artificial Intelligence and Machine Learning \\
{\sl Minor:} Mathematics

\section*{Experience}
%%%%%%%%%%%%%%%%%%%%%%%%%%%%%%%%%%%%%%%%%%%%%%%%%%%%%%%%%%%%%%%%%%%%%%%%%%%
{\bf Research Assistant} \hfill {\it Fall 2011 - present} \\
Johns Hopkins University - Center for Language and Speech Processing \\
{\sl Advised by:} Jason Eisner \\
{\sl Description:} Using novel reinforcement learning algorithms to learn A* search heuristics---the objective being to effective trade-off speed-accuracy in large weighted logic programs such as constituent parsing.

%%%%%%%%%%%%%%%%%%%%%%%%%%%%%%%%%%%%%%%%%%%%%%%%%%%%%%%%%%%%%%%%%%%%%%%%%%%
{\bf Associate Software Engineer} \hfill {\it Fall 2009 - Summer 2011} \\
Information Extraction and Synthesis Laboratory \\
{\sl Advised by:} Andrew McCallum \\
{\sl Description:} Research in Information Extraction and inference in very large factor graphs. 
\begin{itemize}
\item Development of Factorie (\texttt{http://factorie.googlecode.com}), a probabilistic programming framework. Implementation of a competitive Named Entity Recognition and the Loopy Belief Propagation module. 
\item Development of Rexa (\texttt{http://rexa.info}), a scientific literature search engine. My efforts focused on improving the extraction of metadata from documents crawled from the web.
\end{itemize}

%%%%%%%%%%%%%%%%%%%%%%%%%%%%%%%%%%%%%%%%%%%%%%%%%%%%%%%%%%%%%%%%%%%%%%%%%%%
{\bf Research Programmer} \hfill {\it Fall 2008 - Summer 2009} \\
Cognitive Computation Group \texttt{(http://l2r.cs.uiuc.edu/$\sim$cogcomp/)} \\
{\sl Advised by:} Dan Roth \\
{\sl Description:} Research in natural language processing and information extraction. Half of my efforts have been focused on developing a theory and system for reasoning about numeric and temporal quantities in the context of textual entailment, e.g., what makes two quantities equivalent to another? The other half is dedicated to maintaining, developing, and integrating the numerous modules of a large textual entailment System.

%%%%%%%%%%%%%%%%%%%%%%%%%%%%%%%%%%%%%%%%%%%%%%%%%%%%%%%%%%%%%%%%%%%%%%%%%%%
{\bf Project Leader} \hfill {\it Summer 2008} \\
Multimodal Information Access and Synthesis \texttt{(http://mias.uiuc.edu)} \\
{\sl Description:} We built a vertical search engine for events in the Champaign-Urbana, IL area. This search engine aggregates results from a number of event listings and provides a novel search interface on top of an information retrieval engine. The search interface allows users to search for events using natural language queries, such as ``musical performances this week near downtown Champaign'' or ``free all ages concerts in the next month.''

%%%%%%%%%%%%%%%%%%%%%%%%%%%%%%%%%%%%%%%%%%%%%%%%%%%%%%%%%%%%%%%%%%%%%%%%%%%
{\bf Student} \hfill {\it Summer 2007} \\
Multimodal Information Access and Synthesis \texttt{(http://mias.uiuc.edu)} \\
{\sl Description:} We worked on a Content-Based Image Retrieval System.				
				
%%%%%%%%%%%%%%%%%%%%%%%%%%%%%%%%%%%%%%%%%%%%%%%%%%%%%%%%%%%%%%%%%%%%%%%%%%%								
{\bf Intern} \hfill {\it Spring 2002 - Summer 2004} \\
Housing and Development Software \texttt{(http://www.hdsoftware.com/)} \\
{\sl Description:} I worked on the report generation and data entry system of our large data management system used by many government agencies.


\section*{Teaching} 

%%%%%%%%%%%%%%%%%%%%%%%%%%%%%%%%%%%%%%%%%%%%%%%%%%%%%%%%%%%%%%%%%%%%%%%%%%%
{\bf Course Assistant} \hfill {\it Summer 2008} \\
University of Illinois Urbana-Champaign \\
{\sl Course:} CS498FDS - {\sl Foundations of Data Sciences} \\
{\sl Description:} I gave several lectures, wrote assignments and conducted a weekly laboratory sessions where students take what they learned in class and applied it to real problems. Class size was about 30; students where both graduate and undergraduates from all over the US. Topics include: information theory, information retrieval, and machine learning.
				
%%%%%%%%%%%%%%%%%%%%%%%%%%%%%%%%%%%%%%%%%%%%%%%%%%%%%%%%%%%%%%%%%%%%%%%%%%%
{\bf Undergraduate Teaching Assistant} \hfill {\it Spring 2008} \\
University of Illinois Urbana-Champaign \\
{\sl Course:} CS199ch - {\sl Introduction to Programming: A Multimedia             Approach} \\
{\sl Description:} Development of course materials and content for this         pilot course. Hands-on laboratory instruction for class of 10 students.

%%%%%%%%%%%%%%%%%%%%%%%%%%%%%%%%%%%%%%%%%%%%%%%%%%%%%%%%%%%%%%%%%%%%%%%%%%%
{\bf Undergraduate Teaching Assistant} \hfill {\it Fall 2006 - Spring 2008} \\
University of Illinois Urbana-Champaign \\
{\sl Course}: CS173 - {\sl Discrete Structures} \\
{\sl Description:} Lead 2 discussion sections per week ($\approx$ 25 students), I graded homework and exams, proctored exams, held office hours and review sessions (often in attendance of over 150 students).



\section*{Publications} 

%\nocite{*}
%\bibliographystyle{amsplain}
%\bibliography{timv}

Mark Sammons, Vinod Vydiswaran, Tim Vieira, Nikhil Johri, Ming--wei Chang,
Dan Goldwasser, Vivek Srikumar, Gourab Kundu, Yuanchen Tu, Kevin Small,
Joshua Rule, Quang Do, and Dan Roth, \emph{Relation alignment for textual entailment recognition}, Text Analysis Conference (TAC), 2009.



\section*{Software Development} 
{\bf Programming Languages:}
\begin{itemize}
%%%%
\item Python - I was the teaching assistant for a hands on introduction to programming course in Python. I have done extensive experience developing \emph{efficient} scientific programming software including experience with numpy, scipy, weave, Cython, and the Python C API. I have also done a fare share of text processing and CGI programming.
%%%%
\item Web development: JavaScript (prototype.js; script.aculo.us; $<$canvas$>$ element; GreaseMonkey), PHP (MySQL; developing MediaWiki extensions); django web development framework; extensive Python CGI scripting; and various databases including  MongoDB, MySQL and sqlite.
%%%%
\item Scala - Extensive experience developing the factorie (probabilistic programming framework) and implemented applications of factorie such as named entity recognition, metadata extraction (e.g. author, title, citations) from scientific literature (raw PDF documents), and coreference resolution systems.
%%%%
\item Java - Extensive experience in Java in both undergraduate and professional career. Much of the machine learning and information extraction applications I have developed on are written in Java using the Learning-Based Java framework; example applications include numeric and temporal quantity recognition, named entity recognition, noun phrase coreference resolution, and dependency parsing.
%%%%
\item UNIX shell scripting (bash and csh)
%%%%
\item Physical Computing: Arduino micro-controller programming
%%%%
\item I have worked with Mathematica, Emacs Lisp, Prolog, \LaTeX, Scheme (my first language), OCaml, Matlab, MIPS, and Flash/Actionscript on several small personal and class projects.
\end{itemize}

{\bf Technologies:} \\
CGI, XmlHttpRequest/AJAX, apache http server, POSIX system programming primitives (threading, networking, interprocess communication), MongoDB, SQL, xml-rpc, and Google protocol buffers. \\ \vspace{-18pt}
   
{\bf Operating Systems:} \\ 
Ubuntu (linux) is my primary operating system, I have 3+ years of experience developing under Windows XP + Cygwin. \\ 


\section*{Misc}
I am a dual citizen of Brazil and the USA. I am fluent in Portuguese (\emph{Eu falo Portugu\^{e}s}).

\end{document}




