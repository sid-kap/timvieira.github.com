\documentclass[margin]{res}
\usepackage{graphicx}

\setlength{\textwidth}{5.1in}

\newcommand{\centerheader}[1]{\moveleft.5\hoffset\centerline{#1}}

\begin{document}

{
	\moveleft\hoffset\vbox{\hrule width\resumewidth height4pt} 
	{\vskip 0.2in}
  \centerheader{\Large \bf Timothy F. Vieira}
	{\par
		{\vskip 0.1in}
		\centerheader{88 Belchertown Rd \#75}
		\centerheader{Amherst, MA 01002}
		\centerheader{(305)\ 491-3648}
		\centerheader{\texttt{tim.f.vieira@gmail.com}}
		\centerheader{\texttt{http://cs.umass.edu/$\sim$timv/}}
	}
	{\vskip 0.1in}
	\moveleft\hoffset\vbox{\hrule width\resumewidth height 1pt}
}
{\vskip -0.20in}

\begin{resume}
\section{EDUCATION} 
		{\bf Bachelor of Science}, University of Illinois Urbana-Champaign \\
    {\sl Major:} Computer Science, College of Engineering \\
    {\sl Concentration:} Artificial Intelligence and Machine Learning \\
    {\sl Minor:} Mathematics

\section{EXPERIENCE}
				%%%%%%%%%%%%%%%%%%%%%%%%%%%%%%%%%%%%%%%%%%%%%%%%%%%%%%%%%%%%%%%%%%%%%%%%%%%
				{\bf Associate Software Engineer} \hfill {\it Fall 2009 - Present} \\
				Information Extraction and Synthesis Laboratory \\
				{\sl Advised by:} Andrew McCallum \\
        {\sl Description:} Research in Information Extraction and inference in very large factor graphs. 
    \begin{itemize}
      \item Development of Factorie, a probabilistic programming framework. Implementation of a
            state-of-the-art Named Entity Recognition and the Loopy Belief Propagation module. 
      \item Development of Rexa, a scientific literature search engine. Most of my efforts were in
            improving the extraction of metadata from PDF document crawled from the web.
    \end{itemize}

				%%%%%%%%%%%%%%%%%%%%%%%%%%%%%%%%%%%%%%%%%%%%%%%%%%%%%%%%%%%%%%%%%%%%%%%%%%%
				{\bf Research Programmer} \hfill {\it Fall 2008 - Summer 2009} \\
				Cognitive Computation Group \texttt{(http://l2r.cs.uiuc.edu/$\sim$cogcomp/)} \\
				{\sl Advised by:} Dan Roth \\
        {\sl Description:} Research in Natural Language Processing and Information Extraction. Half of my efforts have been focused on developing a theory and system for reasoning about Numeric and Temporal Quantities in the context of Textual Entailment, e.g., what makes two quantities equivalent to another? The other half is dedicated to maintaining, developing, and integrating the numerous modules of a large Textual Entailment System.

				%%%%%%%%%%%%%%%%%%%%%%%%%%%%%%%%%%%%%%%%%%%%%%%%%%%%%%%%%%%%%%%%%%%%%%%%%%%
				{\bf Project Leader} \hfill {\it Summer 2008} \\
				Multimodal Information Access and Synthesis \texttt{(http://mias.uiuc.edu)} \\
        {\sl Description:} We built a vertical search engine for events in the Champaign-Urbana, IL area. This search engine aggregates results from a number of event listings and provides a novel search interface on top of an information retrieval engine. The search interface allows users to search for events using natural language queries, such as ``musical performances this week near downtown Champaign'' or ``free all ages concerts in the next month.''

				%%%%%%%%%%%%%%%%%%%%%%%%%%%%%%%%%%%%%%%%%%%%%%%%%%%%%%%%%%%%%%%%%%%%%%%%%%%
				{\bf Course Assistant} \hfill {\it Summer 2008} \\
        University of Illinois Urbana-Champaign \\
        {\sl Course:} CS498FDS - {\sl Foundations of Data Sciences} \\
				{\sl Description:} I gave several lectures, wrote assignments and conducted a weekly laboratory sessions where students take what they learned in class and applied it to real problems. Class size was about 30; students where both graduate and undergraduates from all over the US. Topics include: information theory, information retrieval, and machine learning.
				
				%%%%%%%%%%%%%%%%%%%%%%%%%%%%%%%%%%%%%%%%%%%%%%%%%%%%%%%%%%%%%%%%%%%%%%%%%%%
				{\bf Undergraduate Teaching Assistant} \hfill {\it Spring 2008} \\
        University of Illinois Urbana-Champaign \\
        {\sl Course:} CS199ch - {\sl Introduction to Programming: A Multimedia Approach} \\
				{\sl Description:} Development of course materials and content for this pilot course. Hands-on laboratory instruction for class of 10 students.

				%%%%%%%%%%%%%%%%%%%%%%%%%%%%%%%%%%%%%%%%%%%%%%%%%%%%%%%%%%%%%%%%%%%%%%%%%%%
				{\bf Student} \hfill {\it Summer 2007} \\
				Multimodal Information Access and Synthesis \texttt{(http://mias.uiuc.edu)} \\
				{\sl Description:} We worked on a Content-Based Image Retrieval System.
				
				%%%%%%%%%%%%%%%%%%%%%%%%%%%%%%%%%%%%%%%%%%%%%%%%%%%%%%%%%%%%%%%%%%%%%%%%%%%
        {\bf Undergraduate Teaching Assistant} \hfill {\it Fall 2006 - Spring 2008} \\
        University of Illinois Urbana-Champaign \\
        {\sl Course}: CS173 - {\sl Discrete Structures} \\
				{\sl Description:}	Lead 2 discussion sections per week ($\approx$ 25 students), I graded homework and exams, proctored exams, held office hours and review sessions (often in attendance of over 150 students).
				
				%%%%%%%%%%%%%%%%%%%%%%%%%%%%%%%%%%%%%%%%%%%%%%%%%%%%%%%%%%%%%%%%%%%%%%%%%%%								
        {\bf Intern} \hfill {\it Spring 2002 - Summer 2004} \\
        Housing and Development Software \texttt{(http://www.hdsoftware.com/)} \\
				{\sl Description:} I worked on the report generation and data entry system of our large data management system.

\section{SKILLS} 
		{\bf Programming Languages:}
		\begin{itemize}
			\item Python - I was the teaching assistant for a hands on introduction to programming course in Python. I've done a lot of CGI scripting, scientific programming (with numpy/scipy), and text processing.
			\item Web development: JavaScript (prototype.js; script.aculo.us; $<$canvas$>$ element; GreaseMonkey), PHP (MySQL; developing MediaWiki extensions); django web development framework; and extensive python CGI scripting.
            \item Scala - Extensive Scala experience in Scala in developing the factorie probabilistic programming framework and implemented factorie applications such as Named Entity Recognition, Information Extraction from scientific literature (PDF documents), and Coreference Resolution systems.
			\item Java - Extensive experience in Java in both undergraduate and professional career. Much of the Machine Learning and Information Extraction applications I have worked on are written in Java using the Learning-Based Java framework; example applications include Numeric and Temporal Quantity Recognition, Named Entity Recognition, and Noun Phrase Coreference Resolution.
			\item UNIX shell scripting (bash and csh)
			\item Physical Computing: Arduino microcontroller programming
			\item I have worked with Mathematica, Emacs Lisp, Prolog, \LaTeX, Scheme (my first language), OCaml, Matlab, MIPS, and Flash/Actionscript on several small personal and class projects.
   \end{itemize}

   {\bf Technologies:} \\
   CGI, apache http server, POSIX system programming primitives (threading, networking, interprocess communication), SQL, xml-rpc, and XmlHttpRequest/AJAX. \\ \vspace{-18pt}
   
   {\bf Operating Systems:}\\ 
   Ubuntu is my primary operating system, I have a significant amount of experience in Windows XP + Cygwin.
 \\ \vspace{-18pt}

   {\bf Misc:} \\ 
   Bilingual: English and Portuguese. 

\end{resume}
\end{document}




